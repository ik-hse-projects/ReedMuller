\def\StripPrefix#1>{}
\def\jobis#1{FF\fi
  \def\predicate{#1}%
  \edef\predicate{\expandafter\StripPrefix\meaning\predicate}%
  \edef\job{\jobname}%
  \ifx\job\predicate
}
\if\jobis{ReedMuller-speaker}
    %% Make more compact
    \PassOptionsToClass{
        articleoptions={10pt,twoside}
    }{beamerswitch}
    \newcommand*{\ArticleSuffix}{-speaker}
    \newenvironment{nonspeaker}{\expandafter\comment}{\expandafter\endcomment}
    \newenvironment{forspeaker}{}{}
\else
    %% For public, not so compact
    \PassOptionsToPackage{outerbars}{changebar}
    \PassOptionsToClass{
        articleoptions={12pt,oneside,a4paper}
    }{beamerswitch}
    \newenvironment{nonspeaker}{}{}
    \newenvironment{forspeaker}{\expandafter\comment}{\expandafter\endcomment}
\fi

\PassOptionsToPackage{table}{xcolor}

\documentclass[beameroptions={aspectratio=169}]{beamerswitch}

\usepackage{fontspec}
\usepackage[english,russian]{babel}
\usepackage[verbose=silent]{microtype}
\usepackage{csquotes}
\usepackage{afterpage}
\usepackage{multicol}
\usepackage[ruled]{algorithm2e}
\usepackage[outputdir=aux]{minted}
\usepackage{mparhack}
\usepackage{mathtools}
\RequirePackage{luatex85}
\usepackage[pdftex,color]{changebar}
\usepackage{unicode-math}
\usepackage{diagbox}
\usepackage{verbatim}
\usepackage{xparse}

\NewDocumentEnvironment{sidebyside}{O{.50} o +m +m}{%
  \noindent\begin{minipage}[t][][t]{#1\linewidth}%
  #3% Content of the first minipage
  \end{minipage}%
  \hfill%
  \noindent\begin{minipage}[t][][t]{\IfValueTF{#2}{#2}{#1}\linewidth}%
  #4% Content of the second minipage
  \end{minipage}\\% newline is important, it allows \hfill to work correctly, try removing it ;)
}

\articlelayout{frametitles=none,maketitle}
\mode<article>{
    \usepackage[
        a4paper,
        inner=30mm,
        outer=40mm,
    ]{geometry}
    \hypersetup{colorlinks=true}
}
\mode<handout>{\setbeameroption{show notes}}
\mode<beamer>{
    \setbeameroption{show notes on second screen = right}
    \AtBeginNote{\vspace{-2pt}\addtolength\leftmargini{-2.5em}}
    \AtEndNote{\addtolength\leftmargini{2.5em}}
}

\usetheme{Berkeley}
%\usecolortheme[rgb={1,0.7137,0}]{stzructure}

\usefonttheme{professionalfonts}
\setmainfont[Ligatures=TeX]{CMU Serif}
\setsansfont[Ligatures=TeX]{CMU Sans Serif}
\setmonofont[Scale=MatchLowercase]{FiraMono}

\setmathfont{latinmodern-math.otf}
\setmathfont[range=\varnothing]{Asana-Math.otf}
\setmathfont[range=\int]{latinmodern-math.otf}

\def\UrlFont{\tt\small}

% https://tex.stackexchange.com/a/1960

\usepackage{graphicx}
\graphicspath{{figures/}}

\setbeamerfont{note page}{size=\footnotesize}
\addtobeamertemplate{note page}{\setbeamerfont{itemize/enumerate subbody}{size=\footnotesize}}{}

\definecolor{disabledColor}{RGB}{255,217,224}
\BeforeBeginEnvironment{frame}{%
  \setbeamercolor{background canvas}{bg=white}%
}
\makeatletter
\define@key{beamerframe}{bg}[gray]{%
  \setbeamercolor{background canvas}{bg=#1}%
}
\makeatother

\title{Код Рида-Маллера}
\author[]{Илья Коннов}
\institute[ВШЭ]{{Факультет компьютерных наук}\and{Высшая Школа Экономики}}
\date{\today}
\logo{\includegraphics[height=0.8cm]{cshse.pdf}}

\newtheorem*{theorem*}{Теорема}
\newtheorem*{lemma*}{Лемма}

\DeclareMathOperator{\RM}{RM}
\DeclareMathOperator{\Eval}{Eval}
\newcommand{\gray}[1]{\textcolor{gray}{#1}}
\newcommand{\pluseq}{\mathrel{+}=}
\newcommand{\minuseq}{\mathrel{-}=}
\newcommand{\asteq}{\mathrel{*}=}
\newcommand{\hl}[2][0-]{\alert<presentation:#1>{#2}}

\newcommand{\splitatcommas}[1]{%
  \begingroup
  \begingroup\lccode`~=`, \lowercase{\endgroup
    \edef~{\mathchar\the\mathcode`, \penalty0 \noexpand\hspace{0pt plus 1em}}%
  }\mathcode`,="8000 #1%
  \endgroup
}
\newcommand{\zerodisplayskips}{%
  \setlength{\abovedisplayskip}{0pt}%
  \setlength{\belowdisplayskip}{0pt}%
  \setlength{\abovedisplayshortskip}{0pt}%
  \setlength{\belowdisplayshortskip}{0pt}}

\makeatletter
\newcommand{\globalsetgeometry}{%
  \global\textheight\textheight
  \global\@colht\textheight
  \global\@colroom\textheight
  \global\vsize\textheight
  \global\headheight\headheight
  \global\topskip\topskip
  \global\headsep\headsep
  \global\topmargin\topmargin
  \global\footskip\footskip
  %% The following are not used in most cases
  \global\textwidth\textwidth
  \global\evensidemargin\evensidemargin
  \global\oddsidemargin\oddsidemargin
  \global\baselineskip\baselineskip
  \global\marginparwidth\marginparwidth
  \global\marginparsep\marginparsep
  \global\columnsep\columnsep
  \global\hoffset\hoffset
  \global\voffset\voffset
}
\makeatother

\mode<article:0>{
    \newcommand{\n}{\\}
    \newcommand{\comm}[1]{\note[item]{#1}}
}
\mode<article>{
    \newcommand{\n}{ }
    \setlength{\changebarsep}{15mm}
    \newcommand{\comm}[1]{\cbstart #1 \cbend}
}

\begin{document}
\uselanguage{russian}

\begin{nonspeaker}
\begin{frame}
    \maketitle

    \comm{
        Существует три различных варианта этого доклада:
        \begin{enumerate}
            \item Краткая презентация, которую несложно рассказать, но может быть сложно понять (\href{https://sldr.xyz/ReedMuller/ReedMuller-trans.pdf}{ReedMuller-trans.pdf}).
                \mode<trans>{\textbf{Вы сейчас читаете именно эту версию, но не можете видеть этот текст.}}
            \item Более длинная презентация с ценными комментариями, дополнительными доказательствами и интересными фактами (\href{https://sldr.xyz/ReedMuller/ReedMuller-slides.pdf}{ReedMuller-slides.pdf}).
                \mode<beamer>{\textbf{Вы сейчас читаете именно эту версию.} Слайды с особенным фоном — не вошедшие в маленькую презентацию.}
            \item Текстовая статья со всем содержимым длинной презентации, комментариями на своих местах, а также бонусным приложением с более подробным описанием алгоритма  (\href{https://sldr.xyz/ReedMuller/ReedMuller-article.pdf}{ReedMuller-article.pdf}).
                \mode<article>{\textbf{Вы сейчас читаете именно эту версию.} Невошедшее в презентацию помечено линями слева, а названия слайдов можно найти справа.}
        \end{enumerate}
        Их все можно посмотреть здесь: \href{https://sldr.xyz/ReedMuller/}{https://sldr.xyz/ReedMuller/}

        \vfill

        По любым вопросам: \href{mailto:r-m@sldr.xyz}{r-m@sldr.xyz} или \href{https://t.me/iliago}{t.me/iliago} или \href{https://vk.com/iliago}{vk.com/iliago}.\\
        \bigskip
    }
\end{frame}

\mode<beamer|article>{
    \begin{frame}[bg=disabledColor]{Содержание}
        \clearpage
        \tableofcontents
    \end{frame}
}
\end{nonspeaker}

\mode<article>{
    \if\jobis{ReedMuller-speaker}
        \newgeometry{
            inner=20mm,
            outer=40mm,
            top=15mm,
            bottom=20mm,
            marginparsep=5mm,
            marginpar=30mm
        }
    \else
        \newgeometry{
            inner=20mm,
            outer=50mm,
            top=20mm,
            bottom=25mm,
            marginparsep=5mm,
            marginpar=40mm
        }
    \fi

    \let\oldframe\frame
    \def\frame{%
        \textcolor{lightgray}{\hrule}%
        \medskip

        \oldframe
    }
    \articlelayout{frametitles=margin}
}

\begin{frame}{Введение}
    Код описан Дэвидом Маллером (автор идеи) и Ирвингом Ридом (автор метода декодирования) в сентябре 1954 года.\n
    Обозначается как $\RM(r, m)$, где $r$ — ранг, а $2^m$ — длина кода. Кодирует сообщения длиной $k = \sum_{i=0}^{r} C_m^i$ при помощи $2^m$ бит. \n
    Традиционно, считается что коды бинарные и работают над битами, т.е. $𝔽₂$.
    
    Соглашение: сложение векторов $u, v ∈ 𝔽_2^n$ будем обозначать как $u ⊕ v = (u_1 + v_1, u_2 + v_2, …, u_n + v_n)$.
\end{frame}

\section{Введение}
\begin{frame}{Булевы функции и многочлен Жегалкина}
    Всякую булеву функцию можно записать при помощи таблицы истинности:
    \[
        \begin{array}{cc | c}
            x & y & f(x, y) \\\hline
            0 & 0 & 1 \\
            0 & 1 & 0 \\
            1 & 0 & 0 \\
            1 & 1 & 0
        \end{array}
    \]

    Или при помощи многочлена Жегалкина:
    \[
        f(x, y) = xy + x + y + 1
    \]
\end{frame}

\begin{frame}{Многочлены Жегалкина}
    В общем случае, многочлены будут иметь следующий вид:
    \[
        f(x_1, x_2, …, x_m) = \sum_{S ⊆ \{1, …, m\}} c_S \prod_{i ∈ S} x_i
    \]

    Например, для $m = 2$: $ f(x_1, x_2) = c_{12} \cdot x_{\{1\}}x_2 + c_{\{2\}} \cdot x_2 + c_{\{1\}} \cdot x_1 + c_{\emptyset} \cdot 1$
    
    Всего $n = 2^m$ коэффициентов для описания каждой функции.
\end{frame}

\begin{frame}{Функции небольшой степени}
    Рассмотрим функции, степень многочленов которых не больше $r$:
    \[
        \{ f(x_1, x_2, …, x_m) \mid \deg f ≤ r \}
    \]
    
    Каждую можно записать следующим образом:
    \[
        f(x_1, x_2, …, x_m) = \sum_{\substack{S ⊆ \{1, …, m\}\\|S| ≤ r}} c_S \prod_{i ∈ S} x_i
    \]
    В каждом произведении используется не больше $r$ переменных.
    
    \comm{Замечу, что при $S = \emptyset$, мы считаем, что $\prod_{i∈S} x_i = 1$, таким образом всегда появляется свободный член.}

    Сколько тогда всего коэффициентов используется?
    \[
        k = C_m^0 + C_m^1 + C_m^2 + … + C_m^r = \sum_{i=0}^r C_m^i
    \]
    
    \comm{
        Если говорить несколько проще, то для составления многочленов мы сложим сначала одночлены ($x + y + z + …$), затем произведения одночленов ($xy + yz + xz + …$) и т.д. вплоть до $r$ множителей (поскольку мы работаем в поле $𝔽₂$, здесь нету $x^2, y^2, z^2$, т.к. $a^2 = a$). Тогда легко видеть, почему $k$ именно такое: мы складываем все возможные перестановки сначала для 0 переменных, потом для одной, двух, и так вплоть до $r$ (не не больше, ведь $\deg f ≤ r$).
    }
\end{frame}

\section{Кодирование}
\begin{frame}{Идея кодирования}
    Пусть каждое сообщение (длины $k$) — коэффициенты многочлена от $m$ переменных степени не больше $r$. \n
    Тогда мы можем его представить при помощи $2^m$ бит, подставив все возможные комбинации значений переменных.\n
    \comm{Их $2^m$, поскольку рассматриваем многочлены только над $𝔽₂$ от $m$ переменных.}
    Таким образом получим таблицу истинности, из которой позднее сможем восстановить исходный многочлен, а вместе с ним и сообщение.

    Зафиксировав в таблице порядок строк, можно выделить \textbf{вектор значений}, который и будет кодом.
    \[
        \begin{array}{cc | c}
            x & y & f(x, y) \\\hline
            0 & 0 & 1 \\
            0 & 1 & 0 \\
            1 & 0 & 0 \\
            1 & 1 & 0
        \end{array}
        \implies
        \Eval(f) = \begin{pmatrix}1 & 0 & 0 & 0\end{pmatrix}
    \]
    \comm{Вектор значений — обозначается $\Eval(f)$ — столбец таблицы истинности, содержащий значения функции. Имеет смысл только при зафиксированном порядке строк в таблице. У меня он везде самый обычный, как в примере выше.}
\end{frame}

\begin{frame}{Пример}\mode<article>{\hypertarget{encoding_example}}
    \comm{Здесь и далее я для краткости и удобства записываю битовые векторы не как $\begin{pmatrix}1 & 0 & 0 & 1\end{pmatrix}$, а как $\mathtt{1001}$ при помощи нескучного шрифта.}

    \begin{nonspeaker}
    \comm{
        Для кодирования очень важно понимать, как именно биты сообщения ставятся в соответствие коэффициентам многочлена. Поэтому давайте введём \textbf{соглашение}:
        если упорядочить элементы множества у каждого коэффициента по возрастанию, то коэффициенты сортируются в лексиографическом порядке:
        $c_{1,2}$ раньше $c_{1,3}$, поскольку $2 < 3$ и $c_{2,3}$ раньше $c_{3,4}$, поскольку $2 < 3$.

        Пример для $m=4$:
        \begin{align*}
            f(x_1, x_2, x_3, x_4)
                & = c_{\{1,2,3,4\}} x₁x₂x₃x₄ \\
                & + \begin{multlined}[t]
                    c_{\{1,2,3\}} x₁x₂x₃ + c_{\{1,2,4\}} x₁x₂x₄ + c_{\{1,3,4\}} x₁x₃x₄ +{} \\
                    + c_{\{2,3,4\}} x₂x₃x₄
                \end{multlined}\\
                & + \begin{multlined}[t]
                    c_{\{1,2\}} x₁x₂ + c_{\{1,3\}} x₁x₃ + c_{\{1,4\}} x₁x₄ + c_{\{2,3\}} x₂x₃ + {}\\
                    + c_{\{2,4\}} x₂x₄ + c_{\{3,4\}} x₃x₄
                \end{multlined}\\
                & + c_{\{1\}} x₁ + c_{\{2\}} x₂ + c_{\{3\}} x₃ + c_{\{4\}} x₄ + c_{\varnothing}
        \end{align*}
        Также можно кодировать множества при помощи битов, используя отношение $x∈A \Longleftrightarrow v_x = 1$ (нумерация битов слева направо, начиная с единицы), где свойство остортированности сохраняется и хорошо видно (но только в пределах группы мнономов одной степени):
        \begin{align*}
            f(x_1, x_2, x_3, x_4)
                &= c_{\mathtt{1111}} x₁x₂x₃x₄ \\
                &+ c_{\mathtt{1110}} x₁x₂x₃ + c_{\mathtt{1101}} x₁x₂x₄ + c_{\mathtt{1011}} x₁x₃x₄ + c_{\mathtt{0111}} x₂x₃x₄ \\
                &+ \begin{multlined}[t]
                    c_{\mathtt{1100}} x₁x₂ + c_{\mathtt{1010}} x₁x₃ + c_{\mathtt{1001}} x₁x₄ + c_{\mathtt{0110}} x₂x₃ + {} \\
                        + c_{\mathtt{0101}} x₂x₄ + c_{\mathtt{0011}} x₃x₄
                \end{multlined}\\
                &+ c_{\mathtt{1000}} x₁ + c_{\mathtt{0100}} x₂ + c_{\mathtt{0010}} x₃ + c_{\mathtt{0001}} x₄ + c_\mathtt{0000}
        \end{align*}
        С помощью этого примера легко увидеть порядок для всех остальных конфигураций кода, если вычеркнуть заведомо невозможные слагаемые (напр., содержащие $x_4$ для $m=3$ или мономы слишком большой степени для $r < 4$).
    
}
    \end{nonspeaker}

    \begin{itemize}
        \item $r = 1$ (степень многочлена), $m = 2$ (переменных).\n Это $\RM(1, 2)$.
        \item Тогда наш многочлен: $f(x₁, x₂) = c_{\{2\}} x₂ + c_{\{1\}} x₁ + c_\emptyset$.
        \item Сообщение: $\mathtt{011}$, тогда $f(x₁, x₂) = 0 + x₁ + 1$.
        \item Подставим всевозможные комбинации:
            \[
                \begin{array}{cc | c}
                    x₁ & x₂ & f(x₁, x₂) \\\hline
                    0  & 0  & 1 \\
                    0  & 1  & 1 \\
                    1  & 0  & 0 \\
                    1  & 1  & 0
                \end{array}
            \]
            \comm{Обратите внимание на то, какой используется порядок переменных в таблице истинности. Очень важно чтобы при кодировании и декодировании было согласие и взаимпонимание касательно того, какому набору переменных соответствует каждая строчка.}
        \item Получили код: $\Eval(f) = \mathtt{1100}$.
    \end{itemize}
\end{frame}

\begin{frame}{Декодирование когда потерь нет}
    \comm{Теперь покажем, как можно декодировать когда потерь нет. Этот пример — продолжение предыдущего.}
    
    \mode<presentation>{\addtolength\leftmargini{-1em}}\begin{itemize}
        \item Мы получили код: $\mathtt{1100}$
        \item \sidebyside{
            Представим таблицу истинности.
        }{\zerodisplayskips\vspace{-1\baselineskip}\[
            \begin{array}{cc | c}
                x₁ & x₂ & f(x₁, x₂) \\\hline
                0 & 0 & 1 \\
                0 & 1 & 1 \\
                1 & 0 & 0 \\
                1 & 1 & 0
            \end{array}
        \]}
        \item \sidebyside{
            Подстановками в \\${f(x₁, x₂) = c_2 x₂ + c_1 x₁ + c_0}$\\ получим СЛАУ.
        }{\zerodisplayskips\vspace{-1\baselineskip}\[
            \left\{\begin{array}{c c c c c c c}
                    & &     & & c_0 &=& 1 \\
                    & & c_2 &+& c_0 &=& 1 \\
                c_1 &+&     & & c_0 &=& 0 \\
                c_1 &+& c_2 &+& c_0 &=& 0
            \end{array}\right.
        \]}
        \item $c_{\{1\}} = 1, c_{\{2\}} = 0, c_\emptyset = 1$, исходное сообщение: \texttt{011}.
    \end{itemize}
\end{frame}

\begin{frame}{Коды $0$-го порядка}
    \comm{Отдельно стоит рассмотреть вариант кода при $r=0$, он нам в будущем пригодится для доказательств.}

    Для случая $\RM(0, m)$ нужна функция от $m$ аргументов, степени не выше $0$.
    \comm{Таких функций существует всего лишь две, поскольку мы можем влиять лишь на свободный член. Все остальные коэффициенты обнуляются из-за требования $\deg f ≤ 0$.}

    \begin{itemize}
        \item $f(x_1, x_2, …, x_m) = 0$
        \item $g(x_1, x_2, …, x_m) = 1$
    \end{itemize}

    \medskip
    Таблица истинности:
    \[
    \begin{array}{r}
        \vphantom{x₁ x₂ … xₘ f(x₁, …, xₘ) g(x₁, …, xₘ)} \\
        2^m \left\{
            \vphantom{
                \begin{array}{c c c  c | c c}
                    0 & 0 & \cdots & 0 & 0 & 1 \\
                    0 & 0 & \cdots & 1 & 0 & 1 \\
                      &   & \ddots &   & \vdots & \vdots \\
                    1 & 1 & \cdots & 1 & 0 & 1
                \end{array}
            }
        \right.
    \end{array}
    \hspace{-1em}
    \begin{array}{c c c  c | c c}
        x₁& x₂& … & xₘ& f(x₁, …, xₘ) & g(x₁, …, xₘ) \\
        \hline
        0 & 0 & \cdots & 0 & 0 & 1 \\
        0 & 0 & \cdots & 1 & 0 & 1 \\
          &   & \ddots &   & \vdots & \vdots \\
        1 & 1 & \cdots & 1 & 0 & 1
    \end{array}
    \]
    \comm{Здесь число строк, как и в любой другой таблице истинности, равно $2^m$, а колонки со значениями никак не зависят от аргументов функций. Получается две колонки – одна с нулями, другая с единицами.}

    Вывод: это $2^m$-кратное повторение символа
    \begin{itemize}
        \item Сообщение $0$ даст код $\underbrace{\mathtt{00…0}}_{2^m}$
        \item Сообщение $1$ даст код $\underbrace{\mathtt{11…1}}_{2^m}$
    \end{itemize}
\end{frame}

\begin{frame}{Коды $m$-го порядка}
    \comm{Есть ещё один тривиальный случай, когда $m = r$.}

    Есть $m$ переменных, и мы рассматриваем многочлены $f∈𝔽_2[x_1,…,x_m] : \deg f ≤ m$, т.е. все возможные.\n
    Для $\RM(m, m)$ мы используем все доступные коэффициенты многочлена для кодирования сообщения.\n
    Тогда нет избыточности: $k = \sum_{i=0}^m C_m^i = 2^m = n$ – длина сообщения равна длине кода.

    \begin{block}{}
        Чем меньше порядок кода $r$, тем больше избыточность.
    \end{block}
\end{frame}

\section{Свойства кода}
\begin{frame}{Доказательство линейности}
    \comm{
        Хотим показать, что этот код является линейным, т.е. что его кодовые слова образуют линейное пространство,
        и у нас есть изоморфизм из пространства сообщений ($𝔽₂^k$) в пространство слов ($𝔽₂^m$).
        
        Для этого необходимо немного формализовать всё описанное раньше.
    }
    
    Пусть $C(x)$ кодирует сообщение $x ∈ 𝔽₂^k$ в код $C(x) ∈ 𝔽₂^m$.
    \[
        C(x) = (p_x(a_i) \mid a_i ∈ 𝔽₂^m)
    \]
    где $p_x(a_i)$ — соответствующий сообщению $x$ многочлен. \comm{Пояснение: перебираем все векторы $a_i$ ($2^m$ штук), подставляем каждый в $p_x$ в качестве переменных и таким образом получаем вектор значений (длины $2^m$). Именно он и называется кодом.}
    
    Причём $p_x$ берёт в качестве своих коэффициентов биты из $x$. Поскольку многочлены степени не выше $r$ образуют линейное пространство, то $p_{(x ⊕ y)} = p_x + p_y$.
    
    \comm{
        Напомню, что базис пространства многочленов выглядит примерно так: $1, x, y, z, xy, yz, xz$ (для трёх переменных, степени не выше 2).

        Чтобы преобразовать сообщение в многочлен, мы берём каждый бит сообщения и умножаем его на соответствующий базисный вектор. Очевидно, такое преобразование будет изоморфизмом. Именно поэтому $p_{(x ⊕ y)} = p_x + p_y$. Обратите внимание, что сообщение $x$ это не просто число ($ℤ_{2^k}$) и мы рассматриваем его биты, а реально вектор битов ($ℤ₂^k$). У него операция сложения побитовая.
    }

    Тогда:
    \[
        C(x ⊕ y)_i = p_{(x⊕y)}(a_i) = p_x(a_i) + p_y(a_i) = C(x)_i + C(y)_i
    \]
    т.е. $∀x, y\quad C(x ⊕ y) = C(x) + C(y)$, ч.т.д.
    
    \comm{
        Здесь я использую запись $C(x)_i$ для $i$-го элемента вектора $C(x)$. Поскольку $i$ произвольное, то и весь вектор получился равен.
        Таким образом, этот код действительно линейный и к нему применимы уже известные теоремы!
    }
\end{frame}

\begin{frame}{Последствия линейности}
    \begin{enumerate}
        \item Существует порождающая матрица $G$.
            \[ C(x) = x_{1×k}G_{k×n} = c_{1×n}\]
            \comm{Так можно кодировать сообщения $x$ в коды $c$. Но искать её мы не будем, обойдёмся одними многочленами, это интереснее.}
        \item Минимальное расстояние будет равно минимальному весу Хемминга среди всех кодов.
            \comm{Вес Хэмминга вектора — количество в нём ненулевых элементов.}
            \[ d = \min_{\substack{c∈C\\c≠0}} w(c) \]
            \comm{Доказательство очень просто: минимальное расстояние — вес разности каких-то двух различных кодов, но разность двух кодов тоже будет кодом, т.к. мы в линейном пространстве. Значит достаточно найти минимальный вес, но не учитывая нулевой вектор, т.к. разность равна нулю тогда и только тогда, когда коды равны.}
        \item Корректирующая способность:
            \[ t = \left\lfloor \frac{d - 1}{2} \right\rfloor \]
    \end{enumerate}
    \comm{Однако мы ещё не знаем как выглядят наши коды (как выглядят таблицы истинности функций степени не больше $r$?). А значит не можем ничего сказать про минимальное расстояние.}
\end{frame}

\subsection<trans:0>{Конструкция Плоткина}

\begin{frame}<trans:0>[bg=disabledColor]{Конструкция Плоткина: многочлены}
    Хотим понять как выглядят кодовые слова.

    \begin{itemize}
        \item Код — вектор значений функции $f(x_1, …, x_m) ∈ \RM(r, m)$, причём $\deg f ≤ r$.
            \comm{Порядок очевидно не больше $r$, потому что это условие для включения в пространство кодов $\RM(r, m)$.}

        \item Разделим функцию по $x_1$: $f(x_1, …, x_m) = g(x_2, …, x_m) + x_1 h(x_2, …, x_m)$.
            \comm{Теперь у нас есть две функции от меньшего числа аргументов. Очевидно, так можно сделать всегда, когда m > 1.}

        \item Заметим, что $\deg f ≤ r$, а значит $\deg g ≤ r$ и $\deg h ≤ r-1$.
    \end{itemize}
\end{frame}

\begin{frame}<trans:0>[bg=disabledColor]{Конструкция Плоткина: таблица истинности}
    \comm{Теперь рассмотрим те же функции, но со стороны их таблиц истинности. Нам же интересны именно коды, а они как раз очень тесно связаны с этими таблицами.}

    Ранее: $f(x_1, …, x_m) = g(x_2, …, x_m) + x_1 h(x_2, …, x_m)$.

    \begin{itemize}
        \item Заметим, что таблица истинности $f$ состоит из двух частей: при $x_1 = 0$ и при $x_1 = 1$.
            \[
                \Eval(f) = \begin{pmatrix}
                    \Eval^{[x_1 = 0]}(f) \\\hline
                    \Eval^{[x_1 = 1]}(f)
                \end{pmatrix}
            \]
            \comm{
                Про обозначения: $\Eval(f)$ — таблица для всей функции (вектор значений, если точнее), $\Eval^{[x_1 = 0]}(f)$ — кусок таблицы при $x_1 = 0$, $\Eval^{[x_1 = 1]}(f)$ — кусок таблицы при $x_1 = 1$. Они нам после этого доказательства больше не понадобятся.
            }

        \item Причём $\Eval^{[x_1 = 0]}(f) = \Eval(g)$, а $\Eval^{[x_1 = 0]}(f) \oplus \Eval^{[x_1 = 1]}(f) = \Eval(h)$.
            \comm{
                Это всё следует из ранее полученного утверждения. Если мы подставим $x_1 = 0$, то останется только $g$ — первое равенство очевидно. Если же мы рассмотрим $\Eval^{[x_1 = 1]}(f)$, то получим $\Eval(g + h)$, но если туда прибавить ещё раз $\Eval(g)$, то останется только $\Eval(h)$ (поскольку $1 + 1 = 0$ в $𝔽₂$) — получили второе равенство.
            }

        \item Таким образом, $\Eval(f) = \left(\Eval(g) \mid \Eval(g) ⊕ \Eval(h)\right)$. \comm{Палочка по центру — конкатенация векторов.}
    \end{itemize}
\end{frame}

\begin{frame}<trans:0>[bg=disabledColor]{Конструкция Плоткина: вывод}
    \comm{Теперь собираем всё это в одно важное утверждение.}

    Если дана $f(x_1, …, x_m)$, причём $\deg f ≤ r$, то можно её разделить:
        \[
            f(x_1, …, x_m) = g(x_2, …, x_m) + x_1h(x_2, …, x_m)
        \]
    \comm{Причём мы уже знаем, что $\deg g ≤ r$ и $\deg h ≤ r-1$, если $\deg f ≤ r$ }

    \medskip

    Также известно, что $\Eval(f) = \left(\Eval(g) \mid \Eval(g) ⊕ \Eval(h)\right)$.

    \medskip

    Заметим, что $\Eval(f)$ – кодовое слово (как и для $g$ и $h$).\newline
    Тогда: \begin{tabular}[t]{l l}
        $ c = \Eval(f) ∈ \RM(r, m) $     & (т.к. $ \deg f ≤ r $) \\
        $ u = \Eval(g) ∈ \RM(r, m-1) $   & (т.к. $ \deg g ≤ r $) \\
        $ v = \Eval(h) ∈ \RM(r-1, m-1) $ & (т.к. $ \deg h ≤ r-1 $) \\
    \end{tabular}

    \comm{
        Напомню, что $\RM(r, m)$ включает в себя \textbf{все} функции (их таблицы истинности, если точнее) от $m$ аргументов и степени не выше $r$. Очевидно, наши годятся.
    }
\end{frame}

\begin{frame}{Конструкция Плоткина}
    \begin{theorem*}
        Для всякого кодового слова $c ∈ \RM(r, m)$ можно найти $u ∈ \RM(r, m-1)$ и $v ∈ \RM(r-1, m-1)$, такие что $c = (u \mid u + v)$.
    \end{theorem*}

    \comm{
        Что здесь важно отметить — оба наших новых кодовых слова $u, v$ получились «меньше», чем исходное $c$.

        Это позволяет, во-первых, устраивать индукцию, чем мы скоро и займёмся.
        Во-вторых, это позволяет легко строить большие порождающие матрицы, но мы этим не будем заниматься.
    }
\end{frame}

\subsection{Минимальное расстояние}

\begin{frame}{Минимальное расстояние}
    Хотим найти минимальное расстояние для кода $\RM(r, m)$
    \[
        d = \min_{c∈C, c≠0} w(c)
    \]

    Предположим, что $d = 2^{m - r}$ и докажем по индукции.

    \textbf{База:} $\RM(0, m)$ — единственный бит повторён $2^m$ раз.
    Очевидно, $w(\underbrace{\mathtt{11…1}}_{2^m}) = 2^m = 2^{m-0} ≥ 2^{m-r}$.

    \comm{
        Случай $\RM(0, m)$ мы разбирали раньше, но я напомню. Здесь длина сообщения равна $k = \sum_{i=0}^r C_m^i = C_m^0 = 1$, а длина кода $n = 2^m$. Причём мы просто берём один бит и повторяем его $2^m$ раз (в таблице истинности).

        Замечу, что не рассматриваю второй случай $w(\mathtt{00…0})$, поскольку он нам не нужен для расчёта минимального расстояния. Вариант с нулевым вектором явно выкидывается, см. определение $d$ выше.
    }

    \textbf{Гипотеза:} Если $v∈\RM(r-1, m-1)$, то $w(v) ≥ 2^{m-r}$.

    \textbf{Шаг:} Хотим доказать для $c ∈ \RM(r, m)$.
    \begin{align*}
    w(c) &\overset{(1)}{=} w((u \mid u ⊕ v))
            \overset{(2)}{=} w(u) + w(u ⊕ v)
        ≥\\&\overset{(3)}{≥} w(u) + \left(w(v) - w(u)\right)
            = w(v)
            \overset{IH}{≥} 2^{m-r} \QED
    \end{align*}

    \comm{
        Теперь немного объяснений.

        Переход (1): используем конструкцию Плоткина, чтобы разбить $c$ на конкатенацию двух кодовых слов поменьше.

        Переход (2): $w((x \mid y)) = w(x) + w(y)$. Вес это всего лишь число ненулевых элементов, поэтому нет разницы как мы будем группировать части вектора.

        Переход (3): $w(u ⊕ v) ≥ w(v) - w(u)$. Если у нас в $v$ стоит $w(v)$ бит, то прибавив к нему $u$, мы сможем изменить (обнулить) не больше $w(u)$ бит. Возможно появится больше единиц, но нас интересует нижняя граница.

        Переход (IH): предположение индукции в чистом виде.
    }
\end{frame}

\subsection{Параметры}

\begin{frame}{Свойства и параметры}
    \comm{Теперь можно подвести итоги исследования свойств.} Для бинарного кода $\RM(r, m)$:
    \begin{itemize}
        \item $r ≤ m$
        \item Длина кода: $2^m$
        \item Длина сообщения: $k = \sum_{i=0}^r C_m^i$
        \item Минимальное расстояние: $d = 2^{m - r}$
        \item Корректирующая способность: $t = 2^{m - r - 1} - 1$\comm{,
             поскольку $t = \left\lfloor\frac{d - 1}{2}\right\rfloor = \left\lfloor\frac{2^{m - r}}{2} - \frac{1}{2}\right\rfloor = \left\lfloor2^{m - r - 1} - 0.5\right\rfloor = 2^{m - r - 1} - 1$
        }
        \item Существует порождающая матрица $G$ для кодирования\comm{, она позволяет делать так: $C(x) = xG$. Но я, как обычно, её избегаю. Рекомендую почитать «\href{http://dha.spb.ru/PDF/ReedMullerExamples.pdf}{Коды Рида-Маллера: Примеры исправления ошибок}», если интересно.}
        \item Проверочная матрица $H$ совпадает с порождающей для $\RM(m - r - 1, m)$\comm{, но это я это доказывать не собираюсь. Однако доказательство можно найти в «\href{https://arxiv.org/pdf/2002.03317.pdf}{Reed-Muller Codes: Theory and Algorithms}», раздел Duality.}
    \end{itemize}
\end{frame}

\begin{frame}{Возможные варианты}
    % r = 0, m = 1, k = 1, n = 2, d = 2, t = 0
    % r = 1, m = 1, k = 2, n = 2, d = 1, t = -0.5

    % r = 0, m = 2, k = 1, n = 4, d = 4, t = 1
    % r = 1, m = 2, k = 3, n = 4, d = 2, t = 0
    % r = 2, m = 2, k = 4, n = 4, d = 1, t = -0.5

    % r = 0, m = 3, k = 1, n = 8, d = 8, t = 3
    % r = 1, m = 3, k = 4, n = 8, d = 4, t = 1
    % r = 2, m = 3, k = 7, n = 8, d = 2, t = 0
    % r = 3, m = 3, k = 8, n = 8, d = 1, t = -0.5

    % r = 0, m = 4, k = 1, n = 16, d = 16, t = 7
    % r = 1, m = 4, k = 5, n = 16, d = 8, t = 3
    % r = 2, m = 4, k = 11, n = 16, d = 4, t = 1
    % r = 3, m = 4, k = 15, n = 16, d = 2, t = 0
    % r = 4, m = 4, k = 16, n = 16, d = 1, t = -0.5


    \[\begin{array}{c || c | c | c | c | c }
        \hbox{\diagbox{$m$}{$r$}} & 0 & 1 & 2 & 3 & 4\\
        \hline\hline
        1 & \cellcolor{yellow!25}\begin{array}{l}k=1\\ n=2\\ t=0\end{array} & % r=0
            \cellcolor{red!25}   \begin{array}{l}k=2\\ n=2\\ t=0\end{array} & % r=1
            — &
            — &
            — \\\hline
        2 &                      \begin{array}{l}k=1\\ n=4\\ t=1\end{array} & % r = 0
            \cellcolor{yellow!25}\begin{array}{l}k=3\\ n=4\\ t=0\end{array} & % r = 1
            \cellcolor{red!25}   \begin{array}{l}k=4\\ n=4\\ t=0\end{array} & % r = 2
            — &
            — \\\hline
        3 &                      \begin{array}{l}k=1\\ n=8\\ t=3\end{array} & % r = 0
                                 \begin{array}{l}k=4\\ n=8\\ t=1\end{array} & % r = 1
            \cellcolor{yellow!25}\begin{array}{l}k=7\\ n=8\\ t=0\end{array} & % r = 2
            \cellcolor{red!25}   \begin{array}{l}k=8\\ n=8\\ t=0\end{array} & % r = 3
            — \\\hline
        4 &                      \begin{array}{l}k=1\\  n=16\\ t=7\end{array} & % r = 0
                                 \begin{array}{l}k=5\\  n=16\\ t=3\end{array} & % r = 1
                                 \begin{array}{l}k=11\\ n=16\\ t=1\end{array} & % r = 2
            \cellcolor{yellow!25}\begin{array}{l}k=15\\ n=16\\ t=0\end{array} & % r = 3
            \cellcolor{red!25}   \begin{array}{l}k=16\\ n=16\\ t=0\end{array}   % r = 4
    \end{array}\]

    \comm{У красных кодов минимальное расстояние $d$ равно единице — они совершенно бесполезны, там количество кодов равно количеству сообщений;
             у желтых кодов $d = 2$ — они могут определить наличие ошибки, но не могут её исправить.
             Для всех остальных кодов $d = 2(t + 1)$.}

    \comm{Напоминание: $k$ — длина сообщения, $n$ — длина кода, а $t$ — количество ошибок, которое код точно сможет исправить.
             Заодно о параметрах кода: $m$ — количество переменных у функции (очень влияет на длину кода), а $r$ — максимальная степень многочлена (очень влияет на длину сообщения, и соотвественно надёжность кода), причём $r ≤ m$. Конечно, таблицу можно продолжать и дальше.}

    \comm{И кстати, случай $m=0, k=0$ (не влез) будет собой представлять колирование единственного бита совершенно без изменений.}
\end{frame}

\section{Декодирование}

\begin{frame}{Как линейный код}
    Этот код является линейным кодом, к нему применимы все обычные (и неэффективные методы):
    \begin{itemize}
        \item Перебор по всему пространству кодовых слов в поисках ближайшего. \comm{Этот способ применим ко всем кодам, но никто в здравом уме им не пользуется.}
        \item С использованием синдромов: $s = rH^T$. \comm{Здесь $s$ — синдром, $r$ — полученное сообщение, $H$ — проверочная матрица. Этот метод обычен для линейных кодов.}
    \end{itemize}

    \comm{Эти способы нужно иметь в виду, но о них было рассказано и без меня, так что я их пропущу.}
\end{frame}

%\mode<trans:0>{
%    \subsubsection{Пара слов о синдромах}
%    \begin{frame}[bg=disabledColor]{Синдромы и как их использовать}
%        \comm{Я не стал включать это в презентацию, но вообще-то говоря метод полезный, так что пусть будет здесь.}
%
%        Пусть у нас в полученном сообщении $r$ есть ошибка $e$. Тогда $r = v + e$, где $v$ — кодовое слово, которое крайне легко можно декодировать. Получается, что $s = rH^T = (v + e)H^T = vH^T + eH^T = eH^T$, поскольку $vH^T = 0$ (есть такое свойство). Мы можем перебрать всевозможные ошибки ($e$), для каждой посчитать синдром и записать всё это в таблицу. Тогда, чтобы восстановить сообщение, нужно посчитать синдром, по таблице найти ошибку и исправить её.
%
%        \comm{Источник: \url{https://ru.wikipedia.org/wiki/Линейный_код}}
%    \end{frame}
%}

\subsection{Алгоритм Рида}

\begin{frame}{Определения}
    \comm{Начать стоит с нескольких определений, без которых алгоритм Рида объяснить не получится.}

    \begin{enumerate}
        \item Пусть $A ⊆ \{1,…,m\}$ для $m∈ℕ$
        \item Подпространство $V_A ⊆ 𝔽₂^m$, которое обнуляет все $v_i$, если $i\not\in A$:
        $\displaystyle
            V_A = \{ v∈𝔽_2^m : v_i = 0\ ∀i\not\in A \}
        $
        \item Аналогично для $V_{\overbar{A}}$, где $\overbar{A} = \{1,…,m\} ∖ A$: 
        $\displaystyle
            V_{\overbar{A}} = \{ v∈𝔽_2^m : v_i = 0\ ∀i\in A \}
        $
    \end{enumerate}
    Пример:
    \begin{itemize}
        \item Пусть $m=3, A = \{1, 2\}$, тогда…
        \item $𝔽₂^m = \{\splitatcommas{\mathtt{000}, \mathtt{001}, \mathtt{010}, \mathtt{011}, \mathtt{100}, \mathtt{101}, \mathtt{110}, \mathtt{111}}\}$ \comm{ — все 8 векторов этого пространства}
        \item $V_A = \{\mathtt{000}, \mathtt{010}, \mathtt{100}, \mathtt{110}\}$ ($v_3 = 0\, ∀v$) \comm{ — обнулилась третья позиция, первые две остались}
        \item $\overbar{A} = \{1,2,3\} ∖ A = \{3\}$
        \item $V_{\overbar{A}} = \{ \mathtt{000}, \mathtt{001} \}$ ($v_1 = v_2 = 0\, ∀v$) \comm{— осталась только третья позиция, остальные обнулились. }
    \end{itemize}
\end{frame}

\begin{frame}{Смежные классы}
    Если фиксировано $V_A ⊆ 𝔽₂^m$, то для каждого $b∈𝔽₂^m$ существует смежный класс $V_A + b$:
    \[
        (V_A + b) = \{ v + b \mid v ∈ V_A \}
    \]

    Утверждается, что если брать $b ∈ V_{\overbar{A}}$, то полученные смежные классы будут все различны (и это будут все смежные классы). \comm{Почему все смежные классы $(V_A + b)$ можно получить именно перебором $b ∈ V_{\overbar{A}}$ можно найти в разделе «Дополнительные доказательства» из пдфки \mode<article>{[\hyperlink{cosets_theorem}{ссылка}]}}
\end{frame}

\begin{frame}{Алгоритм Рида для кода $\RM(r, m)$}
    \comm{Теперь, наконец, сам алгоритм Рида с объяснением, что тут происходит. Почему он именно такой и почему это работает — см. раздел (на русском) «Reed's Algorithm: Unique decoding up to half the code distance» [\ref{ReedsAlgorithm}] в пдфке.}

    Декодирует сообщение $u$, если использовался $\RM(r, m)$.
    Для $\RM(2,2)$: $f(x₁,x₂) = u_{\{1,2\}}x₁x₂ + u_{\{2\}}x₂ + u_{\{1\}}x₁ + u_{\emptyset}$.

    \begin{columns}[onlytextwidth]
        \begin{column}[T]{0.7\textwidth}
            \mode<presentation>{
                \fontsize{9pt}{10pt}
                \RestyleAlgo{plain}
                \setlength{\algomargin}{0pt}
                \SetKwFor{While}{while}{}{}
                \SetKwFor{ForEach}{foreach}{}{}
                \SetKwFor{For}{for}{}{}
            }
            \begin{algorithm}[H]
                \DontPrintSemicolon
                \KwData{\hl[1]{\textrm{vector $y = (y_z ∈ 𝔽_2 \mid z ∈ 𝔽_2^m)$}}}
                \For{\hl[2]{$t \gets r$} \hl[6]{\KwTo $0$}} {
                    \ForEach{\hl[3]{$A ⊆ \{1, …, m\}$ with $|A| = t$}}{
                        \hl[5]{$ c = 0 $} \;
                        \ForEach{$
                            \hl[4]{b ∈ V_{\overbar{A}}}
                        $}{\hl[5]{
                            $c \pluseq \left(\sum\limits_{z ∈\,\hl[4]{(V_A + b)}} y_z\right) \bmod 2$ \;
                        }}
                        $\hl[3]{u_A } \gets \mathbf{1} \hl[5]{\left[c ≥ 2^{m - t - 1}\right]}$\;
                    }
                    \hl[6]{$
                        y \minuseq \Eval\left(
                            \sum\limits_{\substack{A ⊆ \{1,…,m\}\\|A| = t}}
                                u_A \prod_{i∈A}x_i
                        \right)
                    $}\;
                }
            \end{algorithm}
        \end{column}%
        \hspace{-0.10\textwidth}%
        \begin{column}[T]{0.40\textwidth}

            \only<presentation:1>{%
                На вход поступает бинарный вектор $y$ длины $2^m$.
                Это вектор значений функции, возможно с ошибками (но их не больше, чем $t = 2^{m - r - 1} - 1$).
                \comm{Цель — восстановить все коэффициенты при многочлене вида $f(x₁,…,xₘ) = u_\varnothing + u_{1}x_1 + x_2x_2 + … + u_{1,2,…,r}x_{1,2,…,r}$, где $\deg f ≤ r$. Обратите внимание, что для индексов при $u$ используются подмножества $A⊆\{1,…,m\}, |A|≤r$, причём каждый $u_A$ умножается на моном $\prod_{i∈A}x_i$.}
            }

            \only<presentation:2>{%
                Будем восстанавливать сначала коэффициенты $u_A$ при старших степенях, потом поменьше и так пока не восстановим их все. Начинаем с $t=r$.
            }
            \only<presentation:3>{%
                Хотим восстановить все коэффициенты при мономах степени $t$. Для этого перебираем все $A, |A| = t$ и для каждого восстанавливаем коэффициент $u_A$ при $x_{A₁}x_{A₂}…x_{A_t}$.
            }

            \only<presentation:4>{%
                Чтобы восстановить коэффициент, нужно перебрать все смежные классы вида $(V_A + b)$:
                {\zerodisplayskips
                    \begin{align*}
                        V_A = \{& v∈𝔽₂^m \\
                            :\, & v_i = 0\,∀i\not\in A \} \\
                        b ∈ \{& v∈𝔽₂^m \\
                            :\, & v_i = 0\,∀i∈A \}
                    \end{align*}
                }
            }

            \only<presentation:5>{%
                Считаем количество ($c$) смежных классов, в которых\n $\sum\limits_{z ∈ (V_A + b)} y_z = 1 \pmod{2}$. \comm{Если это количество больше порогового значения, то считаем, что $u_A = 1$, иначе же $u_A = 0$.}
                Пороговое значение ($2^{m - t - 1}$) здесь — половина от числа смежных классов.
                Таким образом, если большинство сумм дало $1$, то $u_A = 1$, иначе $u_A = 0$.
            }

            \only<presentation:6>{%
                Затем мы вычитаем из $y$ (вектор значений функции) всё найденное на этой итерации, после чего переходим к мономам меньшей степени.\n
                Повторять до восстановления всех коэффициентов.
            }
        \end{column}
    \end{columns}
\end{frame}

\subsubsection{Пример}

\begin{frame}[t]{Пример}
    Ранее: $\mathtt{011}$ кодируется как $\mathtt{1100}$ при помощи $\RM(1, 2)$ \mode<article>{(см. \hyperlink{encoding_example}{самый первый пример}).}
    \uncover<presentation:2->{
        Положим
        $y_{\mathtt{00}} = 1,
        y_{\mathtt{01}} = 1,
        y_{\mathtt{10}} = 0,
        y_{\mathtt{11}} = 0$
        \mode<article>{— именно так, поскольку $\mathtt{1100}$ — вектор значений, который мы сейчас распаковываем обратно в таблицу истинности. В индексе при $y$ находится вектор значений переменных, а его ($y$) значение — значение функции при этих аргументах.}

        Здесь $m = 2$, значит $A ⊆ \{1, 2\}$. Причём $r = 1$, т.е. $|A| ≤ 1$.
    }

    \mode<presentation>{\smallskip\hrule}
    \bigskip

    \only<presentation:1>{
        \comm{Как происходит кодирование, схематически:}
        \[
            \mathtt{101}
            \leadsto \left(f(x₁, x₂) = x₁ + 1\right)
            \leadsto \begin{array}{|c c | c|}
                x₁& x₂& f \\\hline
                0 & 0 & 1 \\
                0 & 1 & 1 \\
                1 & 0 & 0 \\
                1 & 1 & 0
            \end{array}
            \leadsto \begin{aligned}
                y_{\mathtt{00}} &=& 1 \\
                y_{\mathtt{01}} &=& 1 \\
                y_{\mathtt{10}} &=& 0 \\
                y_{\mathtt{11}} &=& 0
            \end{aligned}
            \leadsto \mathtt{1100}
        \]
    }

    \only<presentation:2>{
        \comm{Теперь начинаем декодирование.}

        \noindent Шаг 1/3: $t = 1, A = \{1\}$
        \begin{itemize}
            \item Здесь $V_A = \{\mathtt{00}, \mathtt{10}\}$ \comm{(меняется только первый бит)}, $V_{\overbar{A}} = \{\mathtt{00}, \mathtt{01}\}$ \comm{(первый бит обнулился)}.\n Нужно рассмотреть два смежных класса\comm{ — по одному на каждый вектор из $V_{\overbar{A}}$}.
            \item $(V_A + \mathtt{00}) = \{\mathtt{00}, \mathtt{10}\}$, сумма: $y_{\mathtt{00}} + y_\mathtt{{10}} = 1 + 0 = 1$
            \item $(V_A + \mathtt{01}) = \{\mathtt{01}, \mathtt{11}\}$, сумма: $y_{\mathtt{01}} + y_\mathtt{{11}} = 1 + 0 = 1$
            \item Итого: $u_A = u_{\{1\}} = 1$
        \end{itemize}
    }

    \only<presentation:3>{
        \noindent Шаг 2/3: $t = 1, A = \{2\}$
        \begin{itemize}
            \item Здесь $V_A = \{\mathtt{00}, \mathtt{01}\}$, $V_{\overbar{A}} = \{\mathtt{00}, \mathtt{10}\}$.\n Нужно рассмотреть два смежных класса \comm{ — по одному на каждый вектор из $V_{\overbar{A}}$.}
            \item $(V_A + \mathtt{00}) = \{\mathtt{00}, \mathtt{01}\}$, сумма: $y_{\mathtt{00}} + y_\mathtt{{01}} = 1 + 1 = 0$
            \item $(V_A + \mathtt{10}) = \{\mathtt{10}, \mathtt{11}\}$, сумма: $y_{\mathtt{10}} + y_\mathtt{{11}} = 0 + 0 = 0$
            \item Итого: $u_A = u_{\{2\}} = 0$
        \end{itemize}
    }

    \only<presentation:4>{
        Перед переходом к $t=0$, нужно вычесть из $y$ вектор значений следующей функции:
        \[
            g(x₁, x₂) = u_{\{2\}}x₂ + u_{\{1\}}x₁ = 0x₂ + 1x₁ = x₁
        \]
        \comm{Здесь мы берём все $u$, полученные при $t = 1$, домножаем каждую на соответствущие ей $x$-ы и получаем функцию от $m$ переменных.}

        Вычислим $\Eval(g)$:
        $\displaystyle
        \begin{array}[t]{c c | c}
            x₁ & x₂ & g(x₁, x₂) \\\hline
            0  & 0  & 0 \\
            0  & 1  & 0 \\
            1  & 0  & 1 \\
            1  & 1  & 1
        \end{array}
        $\\
        \comm{Очень важно, чтобы у вас во всех таблицах истинности (в т.ч. той, которая использовалась при кодировании для получения $y$) был одинаковый порядок строк. Иначе чуда не выйдет.}

        Тогда $y \gets y - \Eval(g) = \mathtt{1100} \oplus \mathtt{0011} = \mathtt{1111}$.
        \comm{Полезно заметить, что в $𝔽₂$ сложение и вычитание — одно и то же.}
    }
\end{frame}

\begin{frame}[t]{Продолжение примера: $t = 0$}
    Теперь
    $y_{\mathtt{00}} = 1,
     y_{\mathtt{01}} = 1,
     y_{\mathtt{10}} = 1,
     y_{\mathtt{11}} = 1$

    \mode<presentation>{\smallskip\hrule}
    \bigskip

    \only<presentation:1>{
        Шаг 3/3: $t = 0, A = \varnothing$
        \begin{itemize}
            \item Здесь $V_A = \{\mathtt{00}\}$, но $V_{\overbar{A}} = \{\mathtt{00}, \mathtt{01}, \mathtt{10}, \mathtt{11}\}$.\n Нужно рассмотреть \textbf{четыре} смежных класса.
            \item $(V_A + \mathtt{00}) = \{\mathtt{00}\}$, сумма: $y_{\mathtt{00}} = 1$
            \item $(V_A + \mathtt{01}) = \{\mathtt{01}\}$, сумма: $y_{\mathtt{01}} = 1$
            \item $(V_A + \mathtt{10}) = \{\mathtt{10}\}$, сумма: $y_{\mathtt{10}} = 1$
            \item $(V_A + \mathtt{11}) = \{\mathtt{11}\}$, сумма: $y_{\mathtt{11}} = 1$
            \item Итого: $u_A = u_{\varnothing} = 1$
        \end{itemize}
    }
    \only<presentation:2>{
        Получили $u_{\{2\}} = 0, u_{\{1\}} = 1, u_{\varnothing} = 1$.\n
        Это значит, что исходный многочлен был таков:
        \[
            f(x₁, x₂) = u_{\{2\}} x₂ + u_{\{1\}} x₁ + u_{\varnothing} = \hl{0 + x₁ + 1},
        \]
        а исходное сообщение: $\mathtt{011}$, как и ожидалось.

        \bigskip

        \begin{block}{Время работы}
            Утверждается, что время работы алгоритма — $O(n \log^r n)$, где $n = 2^m$ — длина кода.
        \end{block}
    }
\end{frame}

\mode<article>{
    \afterpage{\restoregeometry\globalsetgeometry}
    \def\frame{\oldframe}
    \articlelayout{frametitles=none}
}

\section{Домашнее задание}
\begin{frame}[t]{Домашнее задание}
    \comm{
        Замечание: каких-либо требований на методы решения нет, но если используете код — приложите его. Различных способов решить существует больше одного.

        Номер варианта можете определять как $1 + \left((5n + 98) \bmod 2\right)$, но главное напишите его и своё имя.

        Для кодирования использовался тот же порядок строк в таблице истинности, что и в остальной презентации; аргументы идут по столбцам слева направо по возрастанию номера. При формировании сообщения, слагаемые сортируются лексиографически, а затем по убыванию степени (см. примеры в презентации).
    }

    \centering\textbf{Вариант 1}

    \begin{enumerate}
        \item Закодировать сообщение: $\mathtt{1001}$. % Ответ: 11110000 для RM(1, 3)
        \item Декодировать код, если ошибок нет: $\mathtt{1010}$, использовался $\RM(1, 2)$. % Ответ: (1,1,0)
        \item Декодировать код, полученный с ошибками: $\mathtt{1101\,1010}$, использовался $\RM(1, 3)$ % Ответ: (0,1,0,1)
    \end{enumerate}

    \bigskip
    \hrule
    \bigskip

    \centering\textbf{Вариант 2}

    \begin{enumerate}
        \item Закодировать сообщение: $\mathtt{0101}$.  % Ответ: 01011010 для RM(1,3)
        \item Декодировать код, если ошибок нет: $\mathtt{0110}$, использовался $\RM(1, 2)$. % Ответ: (0,1,1)
        \item Декодировать код, полученный с ошибками: $\mathtt{1111\,0100}$, использовался $\RM(1, 3)$ % Ответ: (1,0,0,1)
    \end{enumerate}
\end{frame}

\begin{nonspeaker}
\section{Источники}
\begin{frame}<trans:0>[bg=disabledColor]{Источники}
    \begin{enumerate}
        \item \url{https://arxiv.org/pdf/2002.03317.pdf} — великолепный обзор, очень рекомендую.
        \item \url{http://dha.spb.ru/PDF/ReedMullerExamples.pdf} — очень хорошо и подробно, но используется подход через матрицы, а не через полиномы, а это не весело.
        \item \url{https://en.wikipedia.org/wiki/Reed–Muller_code} — кратко, чётко, понятно, но не описано декодирование.
        \item \url{https://ru.bmstu.wiki/Коды\_Рида-Маллера} — в целом всё есть, но написано очень непонятно;
    \end{enumerate}
\end{frame}
\end{nonspeaker}

\appendix
\mode<article>{
    \section{Reed's Algorithm: Unique decoding up to half the code distance}
\label{ReedsAlgorithm}

В этом разделе описывается алгоритм Рида для $\RM(r, m)$. Он исправляет любые ошибки, вес которых не превышает $2^{m-r-1}$, половину минимального расстояния кода.

Для подмножества $A ⊆ \{1,…,m\}$ определим моном $x_A = \prod_{i∈A} x_i$, где $x_i$ — аргументы булевой функции \gray{[напр., $x_{\{1,2\}} = x_1x_2$]}. Также будем использовать $V_A := \{z ∈ 𝔽_2^m : z_i = 0 \,∀i \not\in A \}$ для обозначения подпространства в $𝔽_2^m$ размерности $|A|$, т.е. $V_A$ это подпространство, в котором для всех векторов $z$ зафиксированы биты $z_i = 0$ при $i \not\in A$.
Для подпространства $V_A$ (в пространстве $𝔽_2^m$) существует $2^{m - |A|}$ смежных класса вида $V_A + b := \{z + b \mid z ∈ V_A\}$, где фиксировано $b ∈ F_2^m$ \gray{[доказательство далее]}.
Тогда для любого $A ⊆ \{1,…,m\}$ и $b ∈ 𝔽_2^m$ мы имеем
\[
    \sum_{z∈(V_A + b)} \Eval_z(x_A) = 1,
\]
а для любых $A \not⊆ B$,
\[
    \sum_{z∈(V_A + b)} \Eval_z(x_B) = 0
\]
Эти две суммы над $𝔽_2$ \gray{[т.е. 1 + 1 + 1 = 1]}.
Первая сумма вытекает из того, что $\Eval_z(x_A) = 1$ если и только если $z_i = 1\, ∀i∈A$, причём существует только один такой $z ∈ (V_A + b)$ \gray{[доказательство далее]}.
Для доказательства второй суммы, нужно заметить, что поскольку $A\not⊆B$, то $∃i ∈ A∖B$, а значит бит $z_i$ не влияет на значение $\Eval_z(x_B)$. Отсюда, $\Eval_{z,z_i = 0}(x_B) = \Eval_{z,z_i = 1}(x_B)$, а значит все единички в этой сумме взаимоуничтожатся.

Предположим, что битовый вектор $y = \left(y_z \mid z ∈ 𝔽_2^m\right)$ — зашумлённая версия кодового слова $\Eval(f) ∈ \RM(r, m)$, такого что $y$ и $\Eval(f)$ отличаются не более чем в $2^{m - r - 1}$ позициях. Алгоритм Рида позволяет восстановить исходное кодовое слово из $y$, извлекая коэффициенты полинома $f$. Поскольку $\deg f ≤ r$, мы всегда это можем записать $f = \sum_{A⊆\{1,…,m\},|A|≤r} u_Ax_A$, где $u_A$ — коэффициенты соответствующих мономов. Алгоритм Рида сначала извлекает все коэффициенты при монмах степени $r$, затем при степени $r-1$, и так далее пока не найдёт их все.

Чтобы восстановить коэффициент $u_A$ при $|A| = r$ \gray{[при мономе степени $r$]}, алгоритм Рида вычисляет сумму $\sum_{z∈(V_A + b)} y_z$ для каждого из $2^{m - r}$ смежных классов подпространства $V_A$, а затем выбирает коэффициент большинством голосов\footnote{В оригинале — «performs a majority vote»; я не смог придумать лучшего перевода.} среди этих $2^{m - r}$ сумм. Если там больше единиц, чем нулей, то восстаналиваем $u_A = 1$, иначе $u_A = 0$. Заметим, что если $y = \Eval(f)$, т.е. ошибки нет, то:
\[
    \sum_{s∈(V_A + b)} y_z
    = \sum_{s ∈ (V_A + b)} \Eval_z\left(\sum_{\substack{B ⊆ \{1,…,m\}\\|B|≤r}} u_Bx_B \right)
    = \sum_{\substack{B ⊆ \{1,…,m\}\\|B|≤r}} u_B \sum_{s ∈ (V_A + b)} \Eval_z(x_B).
\]

Из полученных ранее равенств и при условии, что $B ⊆ \{1,…,m\}$ и $|B| ≤ r = |A|$, получаем $\sum_{z∈(V_A + b)}\Eval_z(x_B) = 1$ тогда и только тогда, когда $B = A$ \gray{[из равенства: $A ⊆ B$, из ограничения: $|B| ≤ |A|$]}.
Отсюда $\sum_{z∈(V_A + b)} y_z = u_A$ для всех $2^{m-r}$ смежных классов вида $V_A + b$ если $y = \Eval(f)$.
Поскольку мы допустили, что $y$ и $\Eval(f)$ отличаются не более чем в $2^{m - r - 1}$ позициях, есть меньше чем $2^{m - r - 1}$ смежных классов, в которых $\sum_{z∈(V_A + b)} y_z ≠ u_A$. После голосования большинством среди этих $2^{m-r}$ сумм, мы найдём правильное значение $u_A$.

После вычисления всех коэффициентов при мономах степени $r$, мы можем посчитать:
\[
    y' = y - \Eval\left(\sum_{\substack{B⊆\{1,…,m\}\\|B|=r}} u_B x_B\right).
\]

Это зашумленная версия кодового слова $\Eval(f - \sum_{B⊆\{1,…,m\},|B|=r} u_B x_B) ∈ \RM(r-1, m)$, и количество оошибок в $y'$ меньше чем $2^{m - r - 1}$ из предположения. Тогда мы можем аналогичным образом восстановить все коэффициенты при мономах степени $r-1$ используя $y'$. Повторять эту процедуру пока не будут восстановлены все коэффициенты $f$.

\begin{theorem*}
    При декодировании кода $\RM(r, m)$ для фиксированного $r$ и растущего $m$, алгоритм Рида корректно устраняет любую ошибку с весом Хэмминга не больше $2^{m - r - 1}$ за $O(n \log^r n)$ по времени\gray{, где $n = 2^m$ — длина кода}.
\end{theorem*}
\gray{[в источнике она без доказательства, но вы можете прочитать алгоритм ниже и попытаться доказать это самостоятельно]}

\begin{algorithm}[H]
    \DontPrintSemicolon
    \caption{Reed's algorithm for decoding $\RM(r, m)$}
    \KwData{Parameters $r$ and $m$ of the RM code, and a binary vector $y = (y_z \mid z ∈ 𝔽_2^m)$ of length $n = 2^m$}
    \KwResult{A codeword $c ∈ \RM(r, m)$}

    $t \gets r$\;
    \While{$ t ≥ 0 $}{
        \ForEach{subset $A ⊆ \{1, …, m\}$ with $|A| = t$}{
            Calculate $\sum_{z ∈ (V_A + b)} y_z$ for all the $2^{m-t}$ cosets of $V_A$\;
            $num1 \gets $ number of cosets $(V_A + b)$ such that $\sum_{z∈(V_A + b)} y_z = 1$\;
            $u_A \gets \mathbf{1}\left[num1 ≥ 2^{m - t - 1}\right]$\;
        }
        $y \gets y - \Eval\left(\sum_{A ⊆ \{1,…,m\}, |A| = t} u_A x_A\right)$\;
        $t \gets t - 1$\;
    }
    $c \gets \Eval\left(\sum_{A ⊆ \{1,…,m\}, |A| ≤ r} u_Ax_A\right)$\;
    \KwRet{$c$}\;
\end{algorithm}
\gray{Подсказка: «coset» — смежный класс.}

В оригинале $\mathbf{1}[\cdot]$ описана как «indicator function» (характеристическая функция), но для меня это несёт мало смысла в этом контексте. Впрочем, из доказательства понятно, что здесь должно иметься ввиду:
\[
\mathbf{1}\left[num1 ≥ 2^{m - t - 1}\right] = \begin{cases}
    1,& num1 ≥ 2^{m - t - 1} \\
    0,& num1 < 2^{m - t - 1}
\end{cases}
\]

\subsection{Дополнительные доказательства}

Далее я подробно доказываю некоторые утверждения, которые не были мне совершенно очевидны, и которые я не смог доказать в четыре слова чтобы включить в основной текст.

\begin{lemma*}
    Для подпространства $V_A$ (размерности $|A|$ в пространстве $𝔽_2^m$) существует $2^{m - |A|}$ смежных класса вида $V_A + b := \{z + b \mid z ∈ V_A\}$, где фиксировано $b ∈ F_2^m$.
\end{lemma*}
\begin{proof}
    Из теоремы Лагранжа, известно что $|G| = |H|\cdot[G : H]$, где $H ⊆ G$, а $[G:H]$ — число различных смежных классов. В нашем случае, $H = V_A, G = 𝔽_2^m$. Тогда $|V_A| = 2^{\dim V_A} = 2^{|A|}$. Таким образом получаем:
    \[
    [G : H] = \frac{|G|}{|H|} = \frac{|F_2^m|}{|V_A|} = \frac{2^m}{2^{|A|}} = 2^{m - |A|}
    \tag*{\qedhere}
    \]
\end{proof}

\begin{lemma*}
    $\Eval_z(x_A) = 1$ если и только если $z_i = 1\, ∀i∈A$, причём существует только один такой $z ∈ (V_A + b)$.
\end{lemma*}
\begin{proof}
    Во-первых, $\Eval_z(x_A) = \Eval_z(x_{A_1}x_{A_2}…x_{A_k})$ по определению $x_A$. Конечно же, оно будет верно если и только если $x_{A_1} = x_{A_2} = … = x_{A_k} = 1$. Другими словами, $∀i∈A\quad z_i = 1$, если подставить значения вектор $z$ на место переменных $x$. Таким образом, первая часть доказана.

    Напомню определение $V_A$:
    \[
    V_A = \{z ∈ 𝔽₂^m \mid z_i = 0 \,∀i \not\in A\}
    \]

    Теперь докажем существование вектора. Пусть искомый вектор существует и равен $z = v + b, v ∈ V_A$. Требуется, чтобы $z_i = 1\,∀i∈A$. Т.е. $v_i + b_i = 1$, а значит $v_i = 1 - b_i$ (при $i∈A$, конечно). Такой $v$ действительно существует в подпространстве $V_A$, потому что определение никак не ограничивает элементы $v_i, i∈A$.

    Единственность следует из того, что все остальные элементы $v$ обязательно обнуляются по определению $V_A$ ($v_i = 0$, если $i\not\in A$). Теперь можно сказать, что $v_i = \begin{cases}1 + b_i,&i∈A\\0,& i\not\in A\end{cases}$ и никак иначе, из чего получаем единственность искомого $z = v + b$.
\end{proof}

\begin{lemma*}
    Размерность $V_A$ равна $|A|$.
\end{lemma*}
\begin{proof}
    Это почти очевидное утверждение. Если рассмотреть каждый из векторов в $V_A$, то у него могут меняться только те координаты, которые не обнулены, и их ровно $|A|$. Получается по одному базисному вектору на каждый элемент из $|A|$.
\end{proof}

Следующая теорма необходима для эффективной реализации алгоритма Рида на нормальном языке программрования.

\begin{theorem*}\hypertarget{cosets_theorem}
    Пусть $\overbar{A} = \{1,…,m\} ∖ A$.
    Для фиксированного $A$, множество смежных классов $\{ V_A + b \mid b ∈ V_{\overbar{A}} \}$ будет содержать их все, причём все различны.
\end{theorem*}
\begin{proof}
    Здесь используются верхние индексы, никакого возведения в степень.

    Сначала докажем, что все эти смежные классы различны.
    Рассмотрим любые два: $(V_A + b^1)$ и $(V_A + b^2)$, где $b^1, b^2 ∈ V_{\overbar{A}}$ и $b^1 ≠ b^2$.
    Можно сказать, что векторы $b^1$ и $b^2$ отличаются хотя бы в одном бите, назовём его $i$-ым. Причём $i∈\overbar{A}$, поскольку все другие биты в $V_{\overbar{A}}$ обнулены. Покажем, что любые векторы $x∈(V_A + b^1)$ и $y∈(V_A + b^2)$ тоже будут отличаться в $i$-ом бите.
    \[\begin{array}{l l l}
        x = v^1 + b^1 \quad & y = v^2 + b^2 \quad & b^1 ≠ b^2\\
        x_i = v^1_i + b^1_i & y_i = v^2_i + b^2_i & b^1_i ≠ b^2_i
    \end{array}\]
    Заметим, что $v^1_i = v^2_i = 0$, поскольку $v_1, v_2 ∈ V_A$, но $i \not\in A$. Получается, что $x_i = 0 + b^1_i$ и $y_i = 0+b^2_i$, причём $b^1_i ≠ b^2_i$. Таким образом $x≠y$ для любых $x∈(V_A + b^1), y∈(V_A + b^2)$.

    Теперь докажем, что мы перечислили все смежные классы. Как доказано ранее, их всего $2^{m - |A|}$. С другой стороны, $|V_{\overbar{A}}| = 2^{|\overbar{A}|} = 2^{m - |A|}$. Поскольку все элементы множества различны, то оно содержит все смежные классы.
\end{proof}

\newgeometry{margin=11mm}
\thispagestyle{empty}
\subsection{Реализация алгоритма}
Битовые векторы храним как int. Нумеруются справа налево, нулевой элемент на самой правой позиции int.
Тогда: $u + v = \mintinline{python}{u ^ v}$ и $v_i = \mintinline{python}{(v >> i) & 1}$ (нумерация здесь с нуля, $i∈\{0,…,n-1\}$).

Множество $A$ также храним при помощи одного int. Если $i ∈ A$, то $A_i = 1$.
\begin{multicols*}{2}
    \setlength{\columnseprule}{0.2pt}
    \setmonofont{Fira Code}[Contextuals=Alternate,Scale=MatchLowercase]
    \fontsize{9pt}{10pt}

    % Пустые строчки делаются меньше (5pt): https://github.com/gpoore/minted/issues/199
    \makeatletter
        \let\FV@ListProcessLine@NoBreak@Orig\FV@ListProcessLine@NoBreak
        \let\FV@ListProcessLine@Break@Orig\FV@ListProcessLine@Break
        \def\FV@ListProcessLine@NoBreak#1{%
          \ifx\FV@Line\empty
            \hbox{}\vspace{\dimexpr-\baselineskip+5pt}%
          \else
            \FV@ListProcessLine@NoBreak@Orig{#1}%
          \fi}
        \def\FV@ListProcessLine@Break#1{%
          \ifx\FV@Line\empty
            \hbox{}\vspace{\dimexpr-\baselineskip+5pt}%
          \else
            \FV@ListProcessLine@Break@Orig{#1}%
          \fi}
    \makeatother
    \inputminted[mathescape,breaklines,python3,tabsize=2]{python}{ReedMuller.py}
\end{multicols*}
\restoregeometry

}
\end{document}
