\section{Алгоритм Рида: однозначное декодирование в рамках корректирующей способности}

В этом разделе описывается алгоритм Рида для $\RM(r, m)$. Он исправляет любые ошибки, вес которых не превышает $2^{m-r-1}$, половину минимального расстояния кода.

Для подмножества $A ⊆ \{1…m\}$ определим $\overline{A} = \{1…m\} ∖ A$. Обозначим моном $x_A = \prod_{i∈A} x_i$, где $x_i$ — аргументы булевой функции \gray{[напр., $x_{\{1,2\}} = x_1x_2$]}. Также будем использовать $V_A := \{z ∈ 𝔽_2^m : z_i = 0 \,∀i \not\in A \}$ для обозначения подпространства $𝔽_2^m$ размерности $|A|$, т.е. $V_A$ это подпространство, в котором для всех векторов $z$ зафиксированы биты $z_i = 0$ при $i \not\in A$.
Для подпространства $V_A$ (мощности $2^{|A|}$ в пространстве $𝔽_2^m$) существует $2^{m - |A|}$ смежных класса вида $V_A + b := \{z + b \mid z ∈ V_A\}$, где фиксировано $b ∈ F_2^m$.
Тогда для любого $A ⊆ \{1,…,m\}$ и $b ∈ 𝔽_2^m$ мы имеем \[\sum_{z∈(V_A + b)} \Eval_z(x_A) = 1,\] а для любых $A \not⊆ B$, \[\sum_{z∈(V_A + b)} \Eval_z(x_B) = 0\]
Эти две суммы над $𝔽_2$ \gray{[т.е. 1 + 1 + 1 = 1]}.
Первая сумма вытекает из того, что $\Eval_z(x_A) = 1$ если и только если $z_i = 1\, ∀i∈A$, причём существует только один такой $z ∈ (V_A + b)$ \gray{[пояснение: $\Eval_z(x_A) = \Eval_z(x_{A_1}x_{A_2}…x_{A_k})$, что действительно верно только при всех единицах, и в $(V_A + b)$ такой вектор будет один из того, как устроено $V_A$]}.
Для доказательства второй суммы, нужно заметить, что поскольку $A\not⊆B$, то $∃i ∈ A∖B$, а значит бит $z_i$ не влияет на значение $\Eval_z(x_B)$. Отсюда, $\Eval_{z,z_i = 0}(x_B) = \Eval_{z,z_i = 1}(x_B)$, а значит все единички в этой сумме взаимоуничтожатся.

Предположим, что битовый вектор $y = \left(y_z \mid z ∈ 𝔽_2^m\right)$ — зашумлённая версия кодового слова $\Eval(f) ∈ \RM(r, m)$, такого что $y$ и $\Eval(f)$ отличаются не более чем в $2^{m - r - 1}$ позициях. Алгоритм Рида позволяет восстановить исходное кодовое слово из $y$, извлекая коэффициенты полинома $f$. Поскольку $\deg f ≤ r$, мы всегда это можем записать $f = \sum_{A⊆\{1,…,m\},|A|≤r} u_Ax_A$, где $u_A$ — коэффициенты соответствующих мономов. Алгоритм Рида сначала извлекает все коэффициенты при монмах степени $r$, затем при степени $r-1$, и так далее пока не найдёт их все.

Чтобы восстановить коэффициент $u_A$ при $|A| = r$ \gray{[при мономе степени $r$]}, алгоритм Рида вычисляет сумму $\sum_{z∈(V_A + b)} y_z$ для каждого из $2^{m - r}$ смежных классов подпространства $V_A$, а затем выбирает коэффициент большинством голосов среди этих $2^{m - r}$ сумм. Если там больше единиц, чем нулей, то восстаналиваем $u_A = 1$, иначе $u_A = 0$. Заметим, что если $y = \Eval(f)$, т.е. ошибки нет, то:
\[
    \sum_{s∈(V_A + b)} y_z
    = \sum_{s ∈ (V_A + b)} \Eval_z\left(\sum_{\substack{B ⊆ \{1,…,m\}\\|B|≤r}} u_Bx_B \right)
    = \sum_{\substack{B ⊆ \{1,…,m\}\\|B|≤r}} u_B \sum_{s ∈ (V_A + b)} \Eval_z(x_B).
\]

Из полученных ранее равенств и при условии, что $B ⊆ \{1,…,m\}$ и $|B| ≤ r = |A|$, получаем $\sum_{z∈(V_A + b)}\Eval_z(x_B) = 1$ тогда и только тогда, когда $B = A$ \gray{[из равенства: $A ⊆ B$, из ограничения: $|B| ≤ |A|$]}.
Отсюда $\sum_{z∈(V_A + b)} y_z = u_A$ для всех $2^{m-r}$ смежных классов вида $V_A + b$ если $y = \Eval(f)$.
Поскольку мы допустили, что $y$ и $\Eval(f)$ отличаются не более чем в $2^{m - r - 1}$ позициях, есть меньше чем $2^{m - r - 1}$ смежных классов, в которых $\sum_{z∈(V_A + b)} y_z ≠ u_A$. После голосования большинством* среди этих $2^{m-r}$ сумм, мы найдём правильное значение $u_A$.

После вычисления всех коэффициентов при мономах степени $r$, мы можем посчитать:
\[
    y' = y - \Eval\left(\sum_{\substack{B⊆\{1,…,m\}\\|B|=r}} u_B x_B\right).
\]

Это зашумленная версия кодового слова $\Eval(f - \sum_{B⊆\{1,…,m\},|B|=r} u_B x_B) ∈ \RM(r-1, m)$, и количество оошибок в $y'$ меньше чем $2^{m - r - 1}$ из предположения. Тогда мы можем аналогичным образом восстановить все коэффициенты при мономах степени $r-1$ используя $y'$. Повторять эту процедуру пока не будут восстановлены все коэффициенты $f$.

\begin{theorem*}
    При декодировании кода $\RM(r, m)$ для фиксированного $r$ и растущего $m$, алгоритм Рида корректно устраняет любую ошибку с весом Хэмминга не больше $2^{m - r - 1}$ за $O(n \log^r n)$ по времени.
\end{theorem*}

\gray{[в источнике она без доказательства]}

\begin{algorithm}[H]
    \DontPrintSemicolon
    \caption{Reed's algorithm for decoding $\RM(r, m)$}
    \KwData{Parameters $r$ and $m$ of the RM code, and a binary vector $y = (y_z \mid z ∈ F_2^m)$ of length $n = 2^m$}
    \KwResult{A codeword $c ∈ \RM(r, m)$}

    $t \gets r$\;
    \While{$ t ≥ 0 $}{
        \ForEach{subset $A ⊆ \{1, …, m\}$ with $|A| = t$}{
            Calculate $\sum_{z ∈ (V_A + b)} y_z$ for all the $2^{m-t}$ cosets of $V_A$\;
            $num1 \gets $ number of cosets $(V_A + b)$ such that $\sum_{z∈(V_A + b)} y_z = 1$\;
            $u_A \gets \mathbf{1}[num1 ≥ 2^{m - t - 1}]$\;
        }
        $y \gets y - \Eval(\sum_{A ⊆ \{1,…,m\}, |A| = t}) u_A x_A$\;
        $t \gets t - 1$\;
    }
    $c \gets \Eval(\sum_{A ⊆ \{1,…,m\}, |A| ≤ r}) u_Ax_A$\;
    \KwRet{$c$}\;
\end{algorithm}
\gray{Подсказка: «coset» — смежный класс.}
