\section{Reed's Algorithm: Unique decoding up to half the code distance}
\label{ReedsAlgorithm}

В этом разделе описывается алгоритм Рида для $\RM(r, m)$. Он исправляет любые ошибки, вес которых не превышает $2^{m-r-1}$, половину минимального расстояния кода.

Для подмножества $A ⊆ \{1…m\}$ определим обозначим моном $x_A = \prod_{i∈A} x_i$, где $x_i$ — аргументы булевой функции \gray{[напр., $x_{\{1,2\}} = x_1x_2$]}. Также будем использовать $V_A := \{z ∈ 𝔽_2^m : z_i = 0 \,∀i \not\in A \}$ для обозначения подпространства в $𝔽_2^m$ размерности $|A|$, т.е. $V_A$ это подпространство, в котором для всех векторов $z$ зафиксированы биты $z_i = 0$ при $i \not\in A$.
Для подпространства $V_A$ (в пространстве $𝔽_2^m$) существует $2^{m - |A|}$ смежных класса вида $V_A + b := \{z + b \mid z ∈ V_A\}$, где фиксировано $b ∈ F_2^m$ \gray{[доказательство далее]}.
Тогда для любого $A ⊆ \{1,…,m\}$ и $b ∈ 𝔽_2^m$ мы имеем
\[
    \sum_{z∈(V_A + b)} \Eval_z(x_A) = 1,
\]
а для любых $A \not⊆ B$,
\[
    \sum_{z∈(V_A + b)} \Eval_z(x_B) = 0
\]
Эти две суммы над $𝔽_2$ \gray{[т.е. 1 + 1 + 1 = 1]}.
Первая сумма вытекает из того, что $\Eval_z(x_A) = 1$ если и только если $z_i = 1\, ∀i∈A$, причём существует только один такой $z ∈ (V_A + b)$ \gray{[доказательство далее]}.
Для доказательства второй суммы, нужно заметить, что поскольку $A\not⊆B$, то $∃i ∈ A∖B$, а значит бит $z_i$ не влияет на значение $\Eval_z(x_B)$. Отсюда, $\Eval_{z,z_i = 0}(x_B) = \Eval_{z,z_i = 1}(x_B)$, а значит все единички в этой сумме взаимоуничтожатся.

Предположим, что битовый вектор $y = \left(y_z \mid z ∈ 𝔽_2^m\right)$ — зашумлённая версия кодового слова $\Eval(f) ∈ \RM(r, m)$, такого что $y$ и $\Eval(f)$ отличаются не более чем в $2^{m - r - 1}$ позициях. Алгоритм Рида позволяет восстановить исходное кодовое слово из $y$, извлекая коэффициенты полинома $f$. Поскольку $\deg f ≤ r$, мы всегда это можем записать $f = \sum_{A⊆\{1,…,m\},|A|≤r} u_Ax_A$, где $u_A$ — коэффициенты соответствующих мономов. Алгоритм Рида сначала извлекает все коэффициенты при монмах степени $r$, затем при степени $r-1$, и так далее пока не найдёт их все.

Чтобы восстановить коэффициент $u_A$ при $|A| = r$ \gray{[при мономе степени $r$]}, алгоритм Рида вычисляет сумму $\sum_{z∈(V_A + b)} y_z$ для каждого из $2^{m - r}$ смежных классов подпространства $V_A$, а затем выбирает коэффициент большинством голосов\footnote{В оригинале — «performs a majority vote»; я не смог придумать лучшего перевода.} среди этих $2^{m - r}$ сумм. Если там больше единиц, чем нулей, то восстаналиваем $u_A = 1$, иначе $u_A = 0$. Заметим, что если $y = \Eval(f)$, т.е. ошибки нет, то:
\[
    \sum_{s∈(V_A + b)} y_z
    = \sum_{s ∈ (V_A + b)} \Eval_z\left(\sum_{\substack{B ⊆ \{1,…,m\}\\|B|≤r}} u_Bx_B \right)
    = \sum_{\substack{B ⊆ \{1,…,m\}\\|B|≤r}} u_B \sum_{s ∈ (V_A + b)} \Eval_z(x_B).
\]

Из полученных ранее равенств и при условии, что $B ⊆ \{1,…,m\}$ и $|B| ≤ r = |A|$, получаем $\sum_{z∈(V_A + b)}\Eval_z(x_B) = 1$ тогда и только тогда, когда $B = A$ \gray{[из равенства: $A ⊆ B$, из ограничения: $|B| ≤ |A|$]}.
Отсюда $\sum_{z∈(V_A + b)} y_z = u_A$ для всех $2^{m-r}$ смежных классов вида $V_A + b$ если $y = \Eval(f)$.
Поскольку мы допустили, что $y$ и $\Eval(f)$ отличаются не более чем в $2^{m - r - 1}$ позициях, есть меньше чем $2^{m - r - 1}$ смежных классов, в которых $\sum_{z∈(V_A + b)} y_z ≠ u_A$. После голосования большинством среди этих $2^{m-r}$ сумм, мы найдём правильное значение $u_A$.

После вычисления всех коэффициентов при мономах степени $r$, мы можем посчитать:
\[
    y' = y - \Eval\left(\sum_{\substack{B⊆\{1,…,m\}\\|B|=r}} u_B x_B\right).
\]

Это зашумленная версия кодового слова $\Eval(f - \sum_{B⊆\{1,…,m\},|B|=r} u_B x_B) ∈ \RM(r-1, m)$, и количество оошибок в $y'$ меньше чем $2^{m - r - 1}$ из предположения. Тогда мы можем аналогичным образом восстановить все коэффициенты при мономах степени $r-1$ используя $y'$. Повторять эту процедуру пока не будут восстановлены все коэффициенты $f$.

\begin{theorem*}
    При декодировании кода $\RM(r, m)$ для фиксированного $r$ и растущего $m$, алгоритм Рида корректно устраняет любую ошибку с весом Хэмминга не больше $2^{m - r - 1}$ за $O(n \log^r n)$ по времени\gray{, где $n = 2^m$ — длина кода}.
\end{theorem*}
\gray{[в источнике она без доказательства, но вы можете прочитать алгоритм ниже и попытаться доказать это самостоятельно]}

\begin{algorithm}[H]
    \DontPrintSemicolon
    \caption{Reed's algorithm for decoding $\RM(r, m)$}
    \KwData{Parameters $r$ and $m$ of the RM code, and a binary vector $y = (y_z \mid z ∈ 𝔽_2^m)$ of length $n = 2^m$}
    \KwResult{A codeword $c ∈ \RM(r, m)$}

    $t \gets r$\;
    \While{$ t ≥ 0 $}{
        \ForEach{subset $A ⊆ \{1, …, m\}$ with $|A| = t$}{
            Calculate $\sum_{z ∈ (V_A + b)} y_z$ for all the $2^{m-t}$ cosets of $V_A$\;
            $num1 \gets $ number of cosets $(V_A + b)$ such that $\sum_{z∈(V_A + b)} y_z = 1$\;
            $u_A \gets \mathbf{1}\left[num1 ≥ 2^{m - t - 1}\right]$\;
        }
        $y \gets y - \Eval\left(\sum_{A ⊆ \{1,…,m\}, |A| = t} u_A x_A\right)$\;
        $t \gets t - 1$\;
    }
    $c \gets \Eval\left(\sum_{A ⊆ \{1,…,m\}, |A| ≤ r} u_Ax_A\right)$\;
    \KwRet{$c$}\;
\end{algorithm}
\gray{Подсказка: «coset» — смежный класс.}

В оригинале $\mathbf{1}[\cdot]$ описана как «indicator function» (характеристическая функция), но для меня это несёт мало смысла в этом контексте. Впрочем, из доказательства понятно, что здесь должно иметься ввиду:
\[
\mathbf{1}\left[num1 ≥ 2^{m - t - 1}\right] = \begin{cases}
    1,& num1 ≥ 2^{m - t - 1} \\
    0,& num1 < 2^{m - t - 1}
\end{cases}
\]

\subsection{Дополнительные доказательства}

Далее я подробно доказываю некоторые утверждения, которые не были мне совершенно очевидны, и которые я не смог доказать в четыре слова чтобы включить в основной текст.

\begin{lemma*}
    Для подпространства $V_A$ (размерности $|A|$ в пространстве $𝔽_2^m$) существует $2^{m - |A|}$ смежных класса вида $V_A + b := \{z + b \mid z ∈ V_A\}$, где фиксировано $b ∈ F_2^m$.
\end{lemma*}
\begin{proof}
    Из теоремы Лагранжа, известно что $|G| = |H|\cdot[G : H]$, где $H ⊆ G$, а $[G:H]$ — число различных смежных классов. В нашем случае, $H = V_A, G = 𝔽_2^m$. Тогда $|V_A| = 2^{\dim V_A} = 2^{|A|}$. Таким образом получаем:
    \[
    [G : H] = \frac{|G|}{|H|} = \frac{|F_2^m|}{|V_A|} = \frac{2^m}{2^{|A|}} = 2^{m - |A|}
    \tag*{\qedhere}
    \]
\end{proof}

\begin{lemma*}
    $\Eval_z(x_A) = 1$ если и только если $z_i = 1\, ∀i∈A$, причём существует только один такой $z ∈ (V_A + b)$.
\end{lemma*}
\begin{proof}
    Во-первых, $\Eval_z(x_A) = \Eval_z(x_{A_1}x_{A_2}…x_{A_k})$ по определению $x_A$. Конечно же, оно будет верно если и только если $x_{A_1} = x_{A_2} = … = x_{A_k} = 1$. Другими словами, $∀i∈A\quad z_i = 1$, если подставить значения вектор $z$ на место переменных $x$. Таким образом, первая часть доказана.

    Напомню определение $V_A$:
    \[
    V_A = \{z ∈ 𝔽₂^m \mid z_i = 0 \,∀i \not\in A\}
    \]

    Теперь докажем существование вектора. Пусть искомый вектор существует и равен $z = v + b, v ∈ V_A$. Требуется, чтобы $z_i = 1\,∀i∈A$. Т.е. $v_i + b_i = 1$, а значит $v_i = 1 - b_i$ (при $i∈A$, конечно). Такой $v$ действительно существует в подпространстве $V_A$, потому что определение никак не ограничивает элементы $v_i, i∈A$.

    Единственность следует из того, что все остальные элементы $v$ обязательно обнуляются по определению $V_A$ ($v_i = 0$, если $i\not\in A$). Теперь можно сказать, что $v_i = \begin{cases}1 + b_i,&i∈A\\0,& i\not\in A\end{cases}$ и никак иначе, из чего получаем единственность искомого $z = v + b$.
\end{proof}

\begin{lemma*}
    Размерность $V_A$ равна $|A|$.
\end{lemma*}
\begin{proof}
    Это почти очевидное утверждение. Если рассмотреть каждый из векторов в $V_A$, то у него могут меняться только те координаты, которые не обнулены, и их ровно $|A|$. Получается по одному базисному вектору на каждый элемент из $|A|$.
\end{proof}

Следующая теорма необходима для эффективной реализации алгоритма Рида на нормальном языке программрования.

\begin{theorem*}\hypertarget{cosets_theorem}
    Пусть $\overbar{A} = \{1,…,m\} ∖ A$.
    Для фиксированного $A$, множество смежных классов $\{ V_A + b \mid b ∈ V_{\overbar{A}} \}$ будет содержать их все, причём все различны.
\end{theorem*}
\begin{proof}
    Здесь используются верхние индексы, никакого возведения в степень.

    Сначала докажем, что все эти смежные классы различны.
    Рассмотрим любые два: $(V_A + b^1)$ и $(V_A + b^2)$, где $b^1, b^2 ∈ V_{\overbar{A}}$ и $b^1 ≠ b^2$.
    Можно сказать, что векторы $b^1$ и $b^2$ отличаются хотя бы в одном бите, назовём его $i$-ым. Причём $i∈\overbar{A}$, поскольку все другие биты в $V_{\overbar{A}}$ обнулены. Покажем, что любые векторы $x∈(V_A + b^1)$ и $y∈(V_A + b^2)$ тоже будут отличаться в $i$-ом бите.
    \[\begin{array}{l l l}
        x = v^1 + b^1 \quad & y = v^2 + b^2 \quad & b^1 ≠ b^2\\
        x_i = v^1_i + b^1_i & y_i = v^2_i + b^2_i & b^1_i ≠ b^2_i
    \end{array}\]
    Заметим, что $v^1_i = v^2_i = 0$, поскольку $v_1, v_2 ∈ V_A$, но $i \not\in A$. Получается, что $x_i = 0 + b^1_i$ и $y_i = 0+b^2_i$, причём $b^1_i ≠ b^2_i$. Таким образом $x≠y$ для любых $x∈(V_A + b^1), y∈(V_A + b^2)$.

    Теперь докажем, что мы перечислили все смежные классы. Как доказано ранее, их всего $2^{m - |A|}$. С другой стороны, $|V_{\overbar{A}}| = 2^{|\overbar{A}|} = 2^{m - |A|}$. Поскольку все элементы множества различны, то оно содержит все смежные классы.
\end{proof}

\newgeometry{margin=11mm}
\thispagestyle{empty}
\subsection{Реализация алгоритма}
Битовые векторы храним как int. Нумеруются справа налево, нулевой элемент на самой правой позиции int.
Тогда: $u + v = \mintinline{python}{u ^ v}$ и $v_i = \mintinline{python}{(v >> i) & 1}$ (нумерация здесь с нуля, $i∈\{0,…,n-1\}$).

Множество $A$ также храним при помощи одного int. Если $i ∈ A$, то $A_i = 1$.
\begin{multicols*}{2}
    \setlength{\columnseprule}{0.2pt}
    \setmonofont{Fira Code}[Contextuals=Alternate,Scale=MatchLowercase]
    \fontsize{9pt}{10pt}

    % Пустые строчки делаются меньше (5pt): https://github.com/gpoore/minted/issues/199
    \makeatletter
        \let\FV@ListProcessLine@NoBreak@Orig\FV@ListProcessLine@NoBreak
        \let\FV@ListProcessLine@Break@Orig\FV@ListProcessLine@Break
        \def\FV@ListProcessLine@NoBreak#1{%
          \ifx\FV@Line\empty
            \hbox{}\vspace{\dimexpr-\baselineskip+5pt}%
          \else
            \FV@ListProcessLine@NoBreak@Orig{#1}%
          \fi}
        \def\FV@ListProcessLine@Break#1{%
          \ifx\FV@Line\empty
            \hbox{}\vspace{\dimexpr-\baselineskip+5pt}%
          \else
            \FV@ListProcessLine@Break@Orig{#1}%
          \fi}
    \makeatother
    \inputminted[mathescape,breaklines,python3,tabsize=2]{python}{ReedMuller.py}
\end{multicols*}
\restoregeometry
